%%%%%%%%%%%%%%%%%%%%%%%%%%%%%%%%%%%%%%%%%
%  My documentation report
%  Objetive: Explain what I did and how, so someone can continue with the investigation
%
% Important note:
% Chapter heading images should have a 2:1 width:height ratio,
% e.g. 920px width and 460px height.
%
%%%%%%%%%%%%%%%%%%%%%%%%%%%%%%%%%%%%%%%%%


%----------------------------------------------------------------------------------------
%	PACKAGES AND OTHER DOCUMENT CONFIGURATIONS
%----------------------------------------------------------------------------------------

\documentclass[12pt]{article} % Default font size and left-justified equations

\usepackage[top=3cm,bottom=3cm,left=3.2cm,right=3.2cm,headsep=10pt,letterpaper]{geometry} % Page margins

\usepackage{xcolor} % Required for specifying colors by name
\definecolor{ocre}{RGB}{52,177,201} % Define the orange color used for highlighting throughout the book

% Font Settings
\usepackage{avant} % Use the Avantgarde font for headings
%\usepackage{times} % Use the Times font for headings
\usepackage{mathptmx} % Use the Adobe Times Roman as the default text font together with math symbols from the Sym­bol, Chancery and Computer Modern fonts
\usepackage{microtype} % Slightly tweak font spacing for aesthetics
\usepackage[utf8]{inputenc} % Required for including letters with accents
\usepackage[T1]{fontenc} % Use 8-bit encoding that has 256 glyphs

% Bibliography
\usepackage[style=apa,sorting=nyt,sortcites=true,autopunct=true,babel=hyphen,hyperref=true,abbreviate=false,backref=true,backend=biber]{biblatex}

\addbibresource{bibliography.bib} % BibTeX bibliography file
\defbibheading{bibempty}{}

\input{structure} % Insert the commands.tex file which contains the majority of the structure behind the template

%----------------------------------------------------------------------------------------
%	Definitions of new commands
%----------------------------------------------------------------------------------------

\def\R{\mathbb{R}}
\newcommand{\cvx}{convex}
\begin{document}

%----------------------------------------------------------------------------------------
%	TITLE PAGE
%----------------------------------------------------------------------------------------

\begingroup
\thispagestyle{empty}
\AddToShipoutPicture*{\put(0,0){\includegraphics[scale=1.05]{Pictures/covveeeer12.PNG}}} % Image background
\centering{
\vspace*{2.5cm}
\par\normalfont\fontsize{35}{35}\sffamily\selectfont
\textbf{Lecture Notes on Cartesian product topology}\\
{\LARGE Math 6201 - Topology } \\ % Book title
\vspace{-0.3cm}{\Large AY 2024-2025 Term 1} \\
{\Large  James Israel B. Montillano} \vspace{0.2 cm}  % Author name
 } \\~\\
{\normalsize Department of Computer, Information Sciences, and Mathematics} \\
{\small (May 2025)} \\
\endgroup

%----------------------------------------------------------------------------------------
%	COPYRIGHT PAGE
%----------------------------------------------------------------------------------------

%\newpage
%~\vfill
%\thispagestyle{empty}

%%\noindent Copyright \copyright\ 2014 Andrea Hidalgo\\ % Copyright notice

%\noindent \textsc{Summer Research Internship, University of Western Ontario}\\

%\noindent \textsc{github.com/LaurethTeX/Clustering}\\ % URL

%\noindent This research was done under the supervision of Dr. Pauline Barmby with the financial support of the MITACS Globalink Research Internship Award within a total of 12 weeks, from June 16th to September 5th of 2014.\\ % License information

%\noindent \textit{First release, August 2014} % Printing/edition date

%----------------------------------------------------------------------------------------
%	TABLE OF CONTENTS
%----------------------------------------------------------------------------------------

\chapterimage{head1.png} % Table of contents heading image

\pagestyle{empty} % No headers
\newpage
\tableofcontents % Print the table of contents itself

%\cleardoublepage % Forces the first chapter to start on an odd page so it's on the right

\pagestyle{fancy} % Print headers again

%----------------------------------------------------------------------------------------
%	CHAPTER 1
%----------------------------------------------------------------------------------------

\newpage
%I edit nya ni romelyn kay kang cherlyn ni.
\section{Preliminaries}
This section reviews some fundamental concepts needed to understand the basic parts on separation axioms on topology. 

\subsection{Sets}
\begin{definition}[Subset]\parencite{lipschutz1965} A set $A$ is a subset of a set $B$ or, equivalently, $B$ is a superset of $A$, written $A\subset B$ or $B\supset A$ iff each element in $A$ also belongs to $B$; that is, if $x\in A$ implies $x\in B$.
\end{definition}

\begin{definition}[Union]\parencite{lipschutz1965}
The union of two sets $A$ and $B$, denoted by $A\cup B$, is the set of all elements which belong to $A$ or $B$, i.e., $A\cup B = \{x:x\in A \quad \text{or} \quad x\in B\}$     
\end{definition}

\begin{definition}[Intersection]\parencite{lipschutz1965}
The intersection of two sets $A$ and $B$, denoted by $A\cap B$, is the set of elements which belong to both $A$ and $B$, i.e., $A\cap B=\{x:x\in A \quad \text{and} \quad x\in B)$   
\end{definition}


\begin{definition}[Power Sets]\parencite{Dugundji1966}
Let $A$ be any set. Its power set $\mathcal{P}(A)$ is the set of all subsets of $A$.
\end{definition}

\begin{definition}[Countable]\parencite{Dugundji1966}
 A set $A$ is countable if it is finite or equivalent to the set $\mathbb{N}$ of counting
 numbers. If $A\equiv N$, then $A$ is called countably infinite or denumerable.
\end{definition}


\subsection{Functions, or Maps}

\begin{definition}[Map] \parencite{Dugundji1966}
Let $X$ and $Y$ be two sets. A \textit{map} $f: X \rightarrow Y$ \textit{(or function with domain $X$ and range $Y$)} is a subset $f\subset X \times Y$ with the property: for each $x\in X$, there is one, and only one, $y\in Y$ satisfying $(x,y)\in f$.To denote $(x,y)\in f$, we write $y=f(x)$ and say that $y$ is the image of $x$ under $f$.
\end{definition}



\begin{definition}[Surjective function]\parencite{holmes_topology}
Let $f : X \to Y$ be a function. We say that $f$ is surjective if for each $y \in Y$ , there exists $x \in X$ such that $f(x) = y$
\end{definition}

\begin{definition}[Injective function]\parencite{holmes_topology}
Let $f : X \to Y$ be a function. We say that
$f$ is injective if $f(x_1) = f(x_2)$ implies $x_1 = x_2 (x_i \in X)$
\end{definition}


\begin{definition}[Bijective function]\parencite{holmes_topology}
Let \( f : X \to Y \) be a function. We say that \( f \) is bijective if it is both injective and surjective.
\end{definition}

%\begin{example}
%$S_1^n  \{ X | X=X^n ,\lambda(x) \ge 0\}$\\
%A matrix with positive eigenvalues.
%\end{example}

\subsection{Topology}
\begin{definition}[Topology]\parencite{Dugundji1966} 
Let $X$ be a set. A \textit{topology} (or topological structure) in $X$ is a family $\tau $ of subsets of $X$ that satisfies:
\begin{enumerate}
    \item [(1).] Each union of members of $\tau$ is also a member of $\tau$.
    \item [(2).] Each \textit{finite} intersection of members of $\tau$ is also a member of $\tau$.
    \item [(3).] $\emptyset$ and $X$ are members of $\tau$.
\end{enumerate}
\end{definition}

\begin{example}[Discrete topology]
 Let $X$ be any set and 
 $\tau=\mathcal{P}(X)$. Then $\tau$ is a topology on $X$.
 \begin{proof}  ~\\
 \begin{enumerate}
     \item Clearly, $X$, $\varnothing \in \tau$
     \item Since every possible subset of $X$ is included in $\tau$, the union of any combination of these subsets will always result in another subset of $X$, which is already in $\tau$.
     \item Since all subsets are open, the intersection of any finite number of them will still be a subset of $X$, hence in $\tau$.
 \end{enumerate}
 Thus, $\tau$ is a topology on $X$.
 \end{proof}
\end{example}

\begin{example}[Indiscrete topology]
Let $X$ be any set and $\tau=\{X,\varnothing\}$.  Then $\tau$ is a topology on $X$
\begin{proof}  ~\\
\begin{enumerate}
    \item $X \in \tau$ and $\varnothing \in \tau$
    \item $X \cup X=X \cup \varnothing= X \in \tau$\\
     $\varnothing \cup \varnothing = \varnothing \in \tau$
    \item $X \cap X= X \in \tau$\\
    $X \cap \varnothing=\varnothing \cap \varnothing = \varnothing \in \tau$
\end{enumerate}
Thus, $\tau$ is a topology on $X$.
\end{proof}
\end{example}


\begin{definition}[Open sets]\parencite{morris2020topology} 
Let $(X,\tau)$ be a topological space. Then the members of $\tau$ are said to be \emph{open sets}.
\end{definition}

\begin{definition}[Closed sets]\parencite{morris2020topology}
Let $(X,\tau)$ be a topological space. A subset $S$ of $X$ is said to be \emph{closed set} in $(X,\tau)$ if its complement in $X$, namely $X \setminus S$, is open in $(X,\tau)$.
\end{definition}

\begin{definition}[Neighborhood]\parencite{holmes_topology}
Let $x \in X$. Any open set containing $x$ is called a neighborhood of $x$.
\end{definition}

\subsection{$G_{\delta}$}
\begin{definition} \parencite{Dugundji1966}
    A set $G$ is called $G_\delta$ if it is the intersection of at most countably many open sets.
\end{definition}
\subsection{Relativization}
\begin{definition}[Subspace topology]\parencite{Dugundji1966}
 Let $(X,\tau )$ be a topological space and $Y \subset X$. The induced topology $\tau_Y$ on
 $Y$ is $\{Y\cap U : U \in \tau\}$. The pair $(Y,\tau_Y)$ is called a subspace of $(X,\tau )$.
\end{definition}

\subsection{Continuous maps and Homeomorphisms}
\begin{definition}[Continuous maps]\parencite{BellezaCM2025}
Let \( (X, \tau_X) \) and \( (Y, \tau_Y) \) be topological spaces. A map \( f : X \to Y \) is called \emph{continuous} if the inverse image of each open set in \( Y \) is open in \( X \).  
That is, \( f^{-1}\) maps $\tau_Y \to \tau_X$.   
\end{definition}

\begin{example}
 Let $(X,\tau)$ be any topological space and $f : (X,\tau) \to (X,\tau)$ is defined
 by $f(x) = x$ for all $x\in X$. Then f is continuous.
\end{example}
\begin{proof}
To show that \( f \) is continuous, we need to verify that the inverse image of every open set in  
\( Y = (X, \tau) \) is open in \( X = (X, \tau) \).

Let \( O \) be an arbitrary open set in \( Y \). By definition, $f^{-1}(O) = \{ x \in X \mid f(x) \in O \}$.
Since \( f(x) = x \), this simplifies to $f^{-1}(O) = \{ x \in X \mid x \in O \} = O$. Since \( O \) is open in \( Y \), and \( Y \) has the same topology as \( X \), \( O \) is also open in \( X \). Thus, \( f^{-1}(O) = O \) is open in \( X \). Since the inverse image of every open set in \( Y \) is open in \( X \), the identity map \( f \) is continuous.
\end{proof}

\begin{remark}
Not every identity function is continuous. To see this, let \( X = \{1, 2, 3, 4\} \), and define two topologies on \( X \), $\tau_1 = \{\emptyset, \{1\}, \{3\}, \{1,3\}, \{2,3\}, \{1,2,3\}, X\}$, \\ $\tau_2 = \{\emptyset, \{1\}, \{3\}, \{1,3\}, \{2\}, \{1,2,3\}, X\}$. Clearly, \( \tau_1 \) and \( \tau_2 \) are topologies on \( X \). Then \( i : (X, \tau_1) \to (X, \tau_2) \) defined by $i(x) = x$ for all $x \in X$ is not continuous, since there exists \( \{2\} \in \tau_2 \) such that $i^{-1}(\{2\}) = \{2\ \notin \tau_1$.
\end{remark}

\begin{definition}[Homeomorphism]\parencite{Dugundji1966}
    A continuous bijective map $f: X \to Y$, such that $f^{-1}: Y \to X$ is also continuous, is called a homeomorphism (or a bicontinuous bijection) and denoted by $f: X \cong Y$. Two spaces $X, Y$ are homeomorphic, written $X \cong Y$, if there is a homeomorphism $f: X \cong Y$.
\end{definition}

\begin{example}
The identity map $1 : X\to X$ is a homeomorphism.
\end{example}

\begin{example}
Two discrete spaces $X$ and $Y$ (similarly, for indiscrete spaces), are homeomorphic if and only if there is a one-to-one function on $X$ onto $Y$ .
\end{example}
\begin{proof}
In a discrete space, every subset is open. Let \( f : X \to Y \) be a bijection (one-to-one and onto). We will show that \( f \) is a homeomorphism. To show \( f \) is continuous, the preimage of any open set in \( Y \) must be open in \( X \). Since \( Y \) is discrete, every subset of \( Y \) is open. Thus, for any open \( V \subseteq Y \),  \( f^{-1}(V) \subseteq X \). Because \( X \) is discrete, \( f^{-1}(V) \) is open in \( X \). Hence, \( f \) is continuous. Similarly, \( f^{-1} : Y \to X \) is a bijection. For any open \( U \subseteq X \), $(f^{-1})^{-1}(U) = f(U) \subseteq Y$. Since \( X \) is discrete, \( U \) is open, and since \( Y \) is discrete, \( f(U) \) is open in \( Y \).  
Thus, \( f^{-1} \) is continuous. Hence, \( f \) is a bijection, continuous, and its inverse \( f^{-1} \) is also continuous. Therefore, \( f \) is a homeomorphism, and \( X \cong Y \).
\end{proof}

\begin{definition}[Topological Invariant]
    We call any property of spaces a topological invariant if whenever it is true for one space $X$, it is also true for every space homemorphic to $X$
\end{definition}


\subsection{$\sigma$-rings}
\begin{definition}
	A nonempty family $\Sigma \subseteq \mathcal{P}(X)$ is called a $\sigma$-ring if
	\begin{enumerate}
		\item $A \in \Sigma \Rightarrow  A^C \in \Sigma$,
		\item $A_i \in \Sigma$ for $i = 1, 2, \dots \Rightarrow \bigcup_{i=1}^\infty A_i \in \Sigma$.
	\end{enumerate}
\end{definition}

\section{Borel Sets}
A Borel set is any subset of a topological space that can be formed from its open sets (or, equivalently, from closed sets) through the operations of countable union, countable intersection, and relative complement. Borel sets are named after Émile Borel.

\subsection{Generating a Borel Set}
	\begin{theorem}\parencite{Dugundji1966}

	There always exists a unique smallest $\sigma$-ring $\mathcal{B}$ containing the topology $\mathcal{T}$ of $X$. $\mathcal{B}$ is called the family of Borel sets in $X$, and $\aleph(\mathcal{B}) \leq \aleph(\mathcal{T})^*$. Furthermore:
	
	\begin{enumerate}
		\item The countable union, countable intersection, and the difference of Borel sets is a Borel set.
		\item Each $F_\sigma$ and each $G_\delta$ is a Borel set.
	\end{enumerate}
		\end{theorem}
\begin{proof}
	Observing that $\mathcal{P}(X)$ is a $\sigma$-ring containing $\mathcal{T}$, and that the intersection of any family of $\sigma$-rings is also a $\sigma$-ring, we define $\mathcal{B}$ to be the intersection of all $\sigma$-rings containing $\mathcal{T}$. Since only two operations are involved, the estimate of $\aleph(\mathcal{B})$ follows from II, 9.4. To establish (1), we need only verify preservation under intersection, and this follows from
	$$
	\complement \bigcap_{i=1}^\infty B_i = \bigcup_{i=1}^\infty \complement B_i
	$$
	where $B_i \in \mathcal{B}, i \in \mathbb{Z}^+$. (2) is trivial.
\end{proof}



\newpage
\addcontentsline{toc}{chapter}{References}		
	\begingroup
            \printbibliography
        \endgroup
\end{document}